\documentclass[runningheads]{llncs}
%\usepackage{graphics}
%\usepackage{rotate}
\begin{document}
\title{Design and Evaluation of an Edge Concurrency Control Protocol
 for Distributed Graph Databases}
\titlerunning{Design and Evaluation of a Concurrency Protocol}

\author{Paul Ezhilchelvan\inst{1}%\orcidID{0000-0002-6190-5685}
    \and Isi Mitrani\inst{1}%\orcidID{0000-0002-7797-7755}
    \and Jack Waudby\inst{1} \and Jim Webber\inst{2}}
\authorrunning{P. Ezhilchelvan et al.}
\institute{School of Computing, Newcastle University, NE4 5TG, UK \\
    \email{\{paul.ezhilchelvan,isi.mitrani,j.waudby2\}@ncl.ac.uk}
    \and
    Neo4j UK, Union House, 182-194 Union Street, London, SE1 0LH \\
    \email{jim.webber@neo4j.com}
}

\maketitle

\begin{abstract}
A new concurrency control protocol for distributed graph databases is
described. It avoids the introduction of certain types of inconsistencies
by aborting vulnerable transactions. An approximate model that allows the
computation of performance measures, including the fraction of
aborted transactions, is developed. The accuracy of the approximations
is assessed by comparing them with simulations, for a variety of
parameter settings.
\keywords{Graph databases $\cdot$ Reciprocal consistency $\cdot$ Edge-order consistency $\cdot$ Arbitration $\cdot$ Stochastic modelling $\cdot$ Simulation}
\end{abstract}

\section{Introduction}

Existing large-scale distributed data stores such as Google Docs, Dynamo
\cite{dec} and Cassandra \cite{cas} implement an `eventually consistent'
update policy (see \cite{vog}). That is, update requests are processed as
soon as they arrive. In some cases this is a reasonable choice. For a
non-partitioned system there are several solutions to dealing with what is
effectively lag between replicas. However, when a database is partitioned
among several hosts, the eventual consistency approach raises serious
problems, especially when there are explicit or (application) implied
relationships between the data stored in different partitions.

For example, a patient might observe an appointment has been booked in
their timeline on partition A, while the corresponding clinician in
partition B hasn't yet blocked off that slot. Eventual consistency makes
it possible for another patient to book into that slot either overwriting
or double-booking the clinician. While each partition on its own will be
eventually consistent, the system as a whole has violated a constraint.

This is similar in a sense to problems that can occur in traditional
databases with Snapshot Isolation (SI), but unlike SI there are no
mechanisms in eventually consistent databases to detect distributed
constraint violations. For distributed graph databases this is a critical
problem because explicit relationships (edges) routinely span across
partitions, and unless both partitions agree reciprocally on the existence,
direction, and content of the edge then the database has become corrupted.

Another example, by Bailis and Ghodsi \cite{bai} refers to an ATM service
where eventual consistency can allow two users to simultaneously
withdraw more money than their (joint) bank account holds; such
an anomaly, on being detected, is reconciled by invoking exception
handlers. Given that an ATM service is expected to be available
24/7 and that account holders are permitted to access only their own
accounts, the eventually consistent approach is appropriate.

A vast majority of common graph database applications, however, allow
data modified by one (user) transaction to be read by an arbitrary
number of other (users') transactions (see Robinson et al, \cite{rob}).
In such cases, data corrupted
by one transaction and read by subsequent transactions, can lead to
further corruption from which it is impossible to recover.
This process, which was studied at some detail by Ezhilchelvan
et al, \cite{emw}, can in time cause the entire database to become
unusable. That is a situation that is certainly worth avoiding.

In this paper we propose a new update protocol where conflicting updates
are detected and handled. Corruption is thus prevented, but the price
paid for this improvement is that some transactions are aborted. In order
to evaluate just how heavy is that price, we also construct and analyse an
approximate model that allows us to compute the average number of transactions
aborted per unit time and other performance measures.

To simplify the protocol presentation and analysis, we assume that the
hosts are reliable and data items in a database are not replicated.
Provisions for crash-tolerance can be incorporated as an orthogonal
aspect by using well-known techniques (e.g., single server abstraction)
and supportive technologies (e.g., Raft \cite{raft}, Paxos \cite{paxos}).

The problem context and the proposed protocol are described in sections 2 and 3.
The approximate model and its analysis are presented in section 4. Some
numerical and simulation results are reported in section 5, while section 6
outlines the conclusions.

\section{Problem description}

A graph database consists of {\em nodes} representing entities, and {\em edges}
representing relations between them (see \cite{rob}). For example,
node X may represent an entity of type \textsf{Author} and Y an entity of type
\textsf{Book}. X and Y will have an edge between them if they have a relation,
e.g. X is an author of Y.

The popularity of the graph database technology owes much to this simple
structure from which sophisticated models can be easily built and be
efficiently used for query or transaction processing. Examples of
operations performed on a graph database are: finding shortest paths between
two locations in a transport network, performing product recommendations,
looking for cancerous patterns in biological data, etc.

When nodes are connected by an edge, the database stores some reciprocal
information at the origin and destination records of that edge. For example,
if there is an edge from node X to node Y, then node X would have an {\em outgoing}
record {\em wrote} and node Y would have an {\em incoming} record {\em wrote},
which can be interpreted as {\em written by}. Maintaining this reciprocal information
enables an edge to be traversed in either direction.

An edge $e$ is said to be \textit{reciprocally consistent}, if its
origin and destination records, denoted as $e_1$ and $e_2$, at the nodes
that it connects, have mutually consistent, reciprocal entries.

In a distributed graph database, graph data is partitioned and each partition is hosted
by a server in a cluster. Partitioning a graph is non-trivial and even the most optimal
partitioning algorithms (e.g., \cite{hua}, \cite{fir}) seek only to minimise, and cannot
eliminate, the presence of distributed edges. The outgoing and incoming records of a
distributed edge are on different hosts\footnote[1]{This is avoided in {\em edge-partitioned}
graph databases, where all instances of a given edge type reside in the same partition, and nodes
are replicated. Then the problem is to ensure that updates to nodes are consistent across partitions.}.
It has been estimated in \cite{sta}, that a
fraction varying between 25\% and 75\% of all edges would be distributed. Maintaining
reciprocal consistency across a distributed edge is challenging because its $e_1$ and $e_2$
records cannot be updated simultaneously. The time interval that elapses between those updates
permits interference among concurrent transactions.

Suppose, for example, that nodes F and S, referring to a flight and a
seat in an airline database, are stored on hosts $H1$ and $H2$ respectively,
with the edge between them indicating availability. Two transactions, $U$
and $V$, write `S is available in F' and `S is booked in F', respectively.
Each update operation is carried out first on one of the hosts and then,
after a small but non-zero `network delay', on the other host. These two
phases of the update are referred to as `part 1' and `part 2', respectively.
The delay interval between them, $D$, is a random variable which may, in
principle, be unbounded.

Such an implementation, if left uncontrolled, makes possible the
introduction of faults in the edge records. This is illustrated in
Figure 1, which shows three possible conflict scenarios between
transactions $U$ and $V$ (time flows downwards). In case (a),
transaction $U$ performs part 1 of the update on $H1$ at time $t$
and part 2 on $H_2$ at time $t+D$. At some point between $t$
and $t+D$, transaction $V$ performs part 1 on $H_2$, and part 2
on $H_1$ some time later. The result of this occurrence is a
violation of reciprocal consistency: the $H_1$ entry ends up
saying `seat S is booked in F', while the $H_2$ entry says
`seat S is available in F'.

\begin{figure}[!ht]
    \begin{center}
        \setlength{\unitlength}{2000sp}%
        %\setlength{\unitlength}{4144sp}%
        \begin{picture}(9525,3994)(1606,-4580)
        \thinlines
        \put(1981,-1186){\vector( 0,-1){2790}}%
        \put(1981,-1816){\vector( 4,-3){1584}}
        \put(3511,-2266){\vector(-4,-3){1540}}
        %\put(3511,-2266){\vector(-4,-3){1584}}
        \put(3511,-1231){\vector( 0,-1){2745}}
        \put(5581,-1186){\vector( 0,-1){2790}}
        \put(5581,-2221){\vector( 2,-1){1566}}
        \put(7111,-1231){\vector( 0,-1){2745}}
        \put(9406,-1231){\vector( 0,-1){2790}}
        \put(9406,-1861){\vector( 1,-1){1580}}
        %\put(9406,-1861){\vector( 1,-1){1642.500}}
        \put(10960,-1276){\vector( 0,-1){2745}}
        %\put(10936,-1276){\vector( 0,-1){2745}}
        \put(9406,-2581){\vector( 4,-1){1535.294}}
        \put(7120,-1796){\vector(-4,-3){1519.200}}
        \put(1891,-736){H1}%
        \put(1621,-1861){t}%
        \put(3466,-736){H2}%
        \put(3691,-3166){t+D}%
        \put(5491,-736){H1}%
        \put(7066,-736){H2}%
        \put(7291,-3166){t+D}%
        \put(9316,-781){H1}%
        \put(10891,-781){H2}%
        \put(11116,-3000){t+D}%
        %\put(11116,-3661){t+D}%
        \put(5356,-2266){t}%
        \put(5806,-2266){U}%
        \put(9136,-2580){t}%
        %\put(9136,-1861){t}%
        \put(2251,-1951){U}%
        \put(9721,-2086){V}%
        %\put(9721,-2086){A}%
        \put(3220,-2330){V}%
        %\put(3286,-2356){B}%
        \put(6706,-1920){V}%
        %\put(6706,-1996){B}%
        \put(9541,-2581){U}%
        %\put(9541,-2581){B}%
        \put(2611,-4516){(a)}%
        \put(6256,-4471){(b)}%
        \put(10100,-4471){(c)}%
        \end{picture}%
        \caption{Three possible conflict scenarios} \label{f1}
    \end{center}
\end{figure}

A similar conflict is shown in case (b), except that
here part 1 of $V$ is performed on $H_2$ {\em before} time $t$, and
part 2 on $H_1$ after $t$. Finally, there is the possibility (c), where
both transaction traverse the edge in the same direction, but $U$
overtakes $V$ during the network delay. The result of that conflict
is that $H1$ claims `seat S is available in F', while $H_2$ says
`seat S is booked in F'.

In order to prevent such conflicts, only transactions whose distributed
updates are `interference-free', should be allowed to proceed. That is,
if part 1 of an update for a given edge precedes part 1 of another update
for the same edge, then so should part 2, and vice versa. One could, of
course, avoid such conflicts by using a strong consistency mechanism, but
the availability and throughput penalties are generally prohibitive.

A different type of possible conflicts arises when transactions update
more than one edge during their lifetime. For example, suppose that
transactions $A$ and $B$ both update edges $e$ and $e^\prime$, and do so
without interference either among themselves or with other transactions.
It may happen that $e$ is updated by $A$ before $B$, while $e^\prime$
is updated by $B$ before $A$, as illustrated in Figure \ref{eo} (here
time flows from left to right and the conflict-free updates are collapsed
to single instants). Such an occurrence, if allowed, would violate the
property of `edge-order consistency' between transactions.

\begin{figure}[!ht]
    \begin{center}
        \setlength{\unitlength}{2700sp}%
        %\setlength{\unitlength}{4144sp}%
        \begin{picture}(7359,2370)(3364,-4711)
        \thinlines
        \put(3376,-4426){\vector( 1, 0){7335}}%
        \multiput(8866,-2581)(0.00000,-111.81818){17}{\line( 0,-1){ 55.909}}%
        \multiput(5536,-2581)(0.00000,-111.81818){17}{\line( 0,-1){ 55.909}}%
        \put(6211,-3391){\vector( 1, 0){2025}}%
        \multiput(6436,-3391)(0.00000,-108.94737){10}{\line( 0,-1){ 54.474}}%
        \multiput(7696,-3391)(0.00000,-108.94737){10}{\line( 0,-1){ 54.474}}%
        \put(5356,-2581){\vector( 1, 0){4140}}%
        \put(3646,-2626){Transaction A}%
        \put(3691,-3481){Transaction B}%
        \put(5446,-4696){$t1$}%
        \put(6391,-4696){$t2$}%
        \put(7651,-4696){$t3$}%
        \put(8821,-4696){$t4$}%
        \put(10486,-4696){time}%
        \put(8776,-2491){$e^\prime$}%
        \put(6391,-3256){$e$}%
        \put(7651,-3256){$e^\prime$}%
        \put(5446,-2491){$e$}%
        \end{picture}%
        \caption{Edge-order consistency violation} \label{eo}
    \end{center}
\end{figure}

Inconsistencies of this type are to be avoided because they compromise an
important database property known as serializability.

\section{Edge Concurrency Control Protocol}

Our protocol employs two distinct mechanisms, referred to as `collision
detection' and `order arbitration', respectively. These are aimed at enforcing
(i) reciprocal consistency for distributed updates and (ii) edge-order consistency
between transactions. Collision detection is applied at every update and may
trigger an immediate abort. Transactions that survive collision detection may,
if necessary, go to order arbitration. The latter may also result in an abort.

Like many concurrency control protocols in the literature (e.g., \cite{occ},
\cite{warp}), our protocol treats updates as being provisional initially.
They become permanent only if, and when, the transaction that contains them
is allowed to commit.

Provisional updates on a given record can occur only one at a time and each is
time-stamped using the local host's clock. Thus, when a transaction attempts to
update a given record, it can identify all other transactions, called
{\em predecessors} (if any), that have earlier updated that record provisionally.
Read operations receive the latest committed version of a record and ignore any
provisionally updated values.

{\bf Collision detection.} Remember that an update operation by a transaction for
a distributed edge has a part 1, carried out at the first host visited, and part 2,
performed at the second host after a network delay. The corresponding provisionally
updated records are now given labels 1 and 2, respectively, and are associated with
the the id of that transaction. These labels act as `history' meta-data indicating
the host where a transaction started and completed its update. The following rule
is applied:

{\em Cancellation rule}: If, by the time part 2 is performed, a previous
provisional update labeled 1 has been observed, but the corresponding label 2 has
not been observed, then this update is cancelled. In other words, an update is
cancelled if it has observed the start of a previous attempt, but not its completion.

When an update is cancelled, the transaction containing it is aborted and all its
provisional records are erased.

According to the this rule, update $U$ in Figure \ref{f1}, cases (a) and (b), is
cancelled because it observes a predecessor $V$ with label 1 in $H2$, but has not
observed $V$'s label 2 in $H1$. Whether update $V$ is cancelled or not, depends on
whether $U$'s provisional updates remain or have been erased by the time $V$ performs
part 2. In case (c), $U$ is cancelled but $V$ is not, because it does not observe
$U$'s label 1.

{\bf Order arbitration.} The purpose of this mechanism is to detect and prevent
edge-order inconsistencies between transactions. It only applies to transactions
that contain more than one distributed update. Those with a single update that
have not been aborted by collision detection are allowed to commit and depart.

Using the records relating to provisional updates, each multi-update transaction
maintains a `predecessor list' containing all predecessor transactions it encountered
during its provisional updates. If that list is empty when the transaction successfully
completes all its provisional updates, then it commits and departs. Otherwise it
goes to arbitration.

The arbiter is a special service, assumed here to have been implemented
in a dedicated host. A list called the
{\em hit-list} is maintained. It contains transactions which, if allowed to
commit, risk violating edge-order consistency. Transactions arriving for
arbitration join a queue and are served in order of arrival. If the transaction
at the head of the queue is not present in the hit-list, it commits. All
transactions in its predecessor list are added to the hit list if not already
there. If it is in the hit list, it aborts and all its provisional updates are
erased. What has happened in this case is an overtaking: the current transaction
was named as a predecessor by a transaction that committed earlier.

We can informally argue that our approach is correct by considering the edge-order
inconsistency depicted in Fig 2. It can be seen that A will have B in its predecessor
list while updating $e^\prime$, and B will observe A as a predecessor while updating
$e$. Both A and B must approach the arbiter because they update more than one edge and
have a non-empty predecessor list. If the first transaction to be processed by the arbiter
is allowed to commit, the second one will be entered in the hit list and will abort. Thus
only one of A and B, but not both, can commit and edge order inconsistency is always avoided.

Note that this approach to arbitration is pessimistic. It aborts a transaction
as soon as it detects a risk of edge-order violation, even though the actual violation
may not occur. Consequently, some
transactions are aborted unnecessarily, just because they are overtaken by
their successors. To eliminate unwarranted aborts, the arbiter would have
to keep much more detailed information about the updates performed by all
transactions, and would have to do considerably more processing.

We now proceed to the task of evaluating certain performance measures,
such as the average number of transactions that are aborted per unit time,
the offered load at the arbiter, and the average time a transaction
remains in the system. Since the processes involved are rather complex,
such an evaluation will inevitably entail approximations. That, in turn,
will necessitate an assessment of the accuracy of those approximations.

\section{Approximate Model}

We are concerned with updates performed on distributed edges in a
graph database (i.e., edges whose source and destination nodes are
stored on different hosts). These edges are divided into $T$ types,
numbered 1, 2, $\ldots$, $T$. The number of edges of type $i$ is $N_i$,
and the probability that an update operation is aimed at an edge of
type $i$ is $p_i$. All edges of a given type are equally likely to be
addressed, so that the probability of accessing a particular edge of
type $i$ is $p_i/N_i$.

Transactions arrive into the system in a Poisson stream, at the rate
of $\lambda$ per second. Each transaction performs a random number, $K$,
of updates for different distributed edges. The distribution of $K$ is
arbitrary: $P(K=k)=r_k$ ($k=1,2,\ldots$). The average number of updates
per transaction is $\kappa$. Thus, the arrival rate of updates at a
{\em particular} distributed edge of type $i$, $\xi_i$, is equal to
\begin{equation} \label{xi}
\xi_i = \frac{\kappa\lambda p_i}{N_i}\;\;;\;\;i=1,2,\ldots,T\;.
\end{equation}

The first approximation is to assume that the arrival process of updates
for a particular edge of type $i$ is Poisson with rate $\xi_i$.

We wish to estimate the probability, $u_i$, that an update, $U$, for an edge
of type $i$, is cancelled due to a collision with another update, $V$, for
the same edge. That is, either $V$ arrives in the opposite host during the
network delay of $U$ (Figure \ref{f1}, case (a)), or $U$ arrives in the
opposite host during the network delay of $V$ (case (b)), or $U$ arrives in
the same host during the network delay of $V$ {\em and} its network delay
completes before that of $V$ (case (c)).

Assume that the network delays are i.i.d random variables distributed
exponentially with parameter $\delta$ (mean $1/\delta$). This may or may not
be an approximation.

Updates for a particular edge of type $i$ arrive in a particular one of the
two hosts involved at rate $\xi_i/2$. Moreover, a given network delay
completes before another with probability 1/2. Hence, we can estimate
the probabilities of cases (a), (b), and (c), $u_i^{(a)}$, $u_i^{(b)}$ and
$u_i^{(c)}$, as
\begin{equation} \label{uia}
u_i^{(a)} = u_i^{(b)} = \frac{\xi_i}{\xi_i + 2\delta}\;\;;\;\;
u_i^{(c)} = \small{\frac{1}{2}} \frac{\xi_i}{\xi_i + 2\delta}
\;\;;\;\;i=1,2,\ldots,T\;,
\end{equation}
where $\xi_i$ is given by (\ref{xi}) and $\delta$ is the parameter of the
network delay. The overall probability, $u_i$, that at least one of those
events will happen, is
\begin{equation} \label{ui}
u_i = 1 - (1-u_i^{(a)})(1-u_i^{(b)})(1-u_i^{(c)})
\approx \frac{2.5\xi_i}{\xi_i + 2\delta}
\;\;;\;\;i=1,2,\ldots,T\;.
\end{equation}
The last approximation in the right-hand side holds when the rate $\xi_i$
is small compared to $\delta$.

The unconditional probability, $u$, that an arbitrary update is cancelled
by the collision detection mechanism, is given by
\begin{equation} \label{u}
u = \sum_{i=1}^T p_iu_i\;.
\end{equation}

The probability, $v_k$, that a transaction containing $k$ updates is aborted
because one of them is involved in a collision, is equal to
\begin{equation} \label{vk}
v_k = 1 - (1-u)^k\;,
\end{equation}
and the unconditional probability, $v$, that a transaction is aborted due to a
collision is given by
\begin{equation} \label{v}
v = \sum_{k=1}^\infty r_kv_k\;.
\end{equation}

Now consider the average run time, $a_k$, of a transaction that contains $k$
update operations. Assume that each update takes time $b$, on the average.
Those times include read operations and computations, as well as network delays.
If the first $j-1$ provisional updates are completed successfully but the $j$-th
update is cancelled as a result of a collision, then the average run time would
be $jb$. Hence, $a_k$ is given by
\begin{equation} \label{ak}
a_k = \sum_{j=1}^k jb(1-u)^{j-1}u + kb(1-u)^k\;,
\end{equation}
where $u$ is given by (\ref{u})

With a little manipulation, this expression can be simplified to
\begin{equation} \label{ak1}
a_k = b\sum_{j=1}^k (1-u)^{j-1} = b\frac{1 - (1-u)^k}{u}\;.
\end{equation}

The unconditional average run time of a transaction, $a$, is equal to
\begin{equation} \label{a}
a = \sum_{k=1}^\infty r_ka_k\;.
\end{equation}

If all provisional updates in a transaction are completed successfully, and if
either there was only one update, or there were no predecessors, then the
transaction commits. Otherwise it goes to the arbiter. The time that a transaction
spends queueing and being served by the arbiter will be referred to as the
`arbitration time'.

Assume (this is another approximation) that each transaction joins the arbiter
queue with probability $\alpha$, independently of the others. That is, the arrival
process is Poisson, with rate $\lambda\alpha$. The arbiter's average service time,
$s$, is a given parameter. Thus the offered load at the arbiter is
$\rho = \lambda\alpha s$.

Treating the arbiter as an $M/M/1$ queue, we estimate the average arbitration time,
$w$, as
\begin{equation} \label{w}
w = \frac{s}{1 - \rho}\;,
\end{equation}
provided that $\rho<1$. If $\rho\geq 1$, then $w =\infty$.
The total average time that a transaction spends in the system is
\begin{equation} \label{W}
W = a + \alpha w\;,
\end{equation}
where $a$ is given by (\ref{a}).

We shall now develop an iterative fixed-point approximation for $\alpha$. Denote
by $d_{j,k}$ the average lifetime of the $j$'th update within a transaction
containing $k$ updates, {\em excluding} any possible arbitration time. By an
argument similar to the one that led to (\ref{ak1}), we obtain
\begin{equation} \label{djk}
d_{j,k} = b\sum_{i=1}^{k+1-j} (1-u)^{i-1} = b\frac{1 - (1-u)^{k+1-j}}{u}\;.
\end{equation}

The lifetime of a randomly chosen update within a transaction containing $k$
updates, $d_k$ (again excluding arbitration), is given by
\begin{equation} \label{dk}
d_k = \frac{1}{k}\sum_{j=1}^k d_{j,k} =
b\frac{(k+1)u + (1-u)^{k+1} -1}{ku^2}\;.
\end{equation}

Hence, the total average time spent in the system by an arbitrary update
({\em including} the arbitration time), $d$, is equal to
\begin{equation} \label{d}
d = \sum_{k=1}^\infty r_k d_k + \alpha w\;,
\end{equation}
where $w$ is given by (\ref{w}).

Now, let $\gamma_i$ be the probability that an update of type $i$ has a
predecessor, i.e. the probability that such an update arrives while a
preceding update for the same edge is still in the system. Assuming that
the update residence times are distributed exponentially with mean $d$
given by (\ref{d}), this can be approximated as
\begin{equation} \label{gi}
\gamma_i = \frac{\xi_id}{1+ \xi_id}\;,
\end{equation}
where $\xi_i$ is given by (\ref{xi}).

The unconditional probability, $\gamma$, that an arbitrary update has a
predecessor, is
\begin{equation} \label{gam}
\gamma = \sum_{i=1}^T p_i\gamma_i\;.
\end{equation}

If a transaction contains $k$ updates, the probability that at least one
of them has a predecessor, $\alpha_k$, is
\begin{equation} \label{alk}
\alpha_k = 1 - (1-\gamma)^k\;.
\end{equation}

Remembering that a transaction goes to the arbiter if it has more than
one update {\em and} all updates avoid collisions {\em and} at least one of them
has a predecessor, we write
\begin{equation} \label{al}
\alpha = \sum_{k=2}^\infty r_k(1-u)^k\alpha_k\;.
\end{equation}

Note that the right-hand side of (\ref{al}) depends on $\alpha$, via (\ref{d})
and (\ref{gi}). In other words, we have a fixed-point equation of the form
\begin{equation} \label{fp}
\alpha = f(\alpha)\;.
\end{equation}
This can be solved by a simple iterative scheme. Start with an initial guess,
$\alpha_0$, say $\alpha_0=0$. At iteration $n$, compute
\begin{equation} \label{iter}
\alpha_n = f(\alpha_{n-1})\;,
\end{equation}
stopping when two consecutive iterations are sufficiently close to each other.

The probability $\alpha$ allows us to evaluate the offered load at the arbiter
queue, and hence estimate the average response time of a transaction, $W$.
Another important performance measure is the rate of aborts, $R$, i.e. the
average number of transactions that are aborted per unit time. Note that a
transaction may be aborted due to a collision, with probability $v$ given by
(\ref{v}), or it may be aborted because it finds itself on the arbiter's hit
list. Denoting the probability of the latter occurrence by $\beta$, we can
write
\begin{equation} \label{R}
R = \lambda [v + (1-v)\beta]\;.
\end{equation}

To find an expression for the probability $\beta$, note that a transaction, $A$,
is aborted by the arbiter if (i) $A$ goes to the arbiter and (ii) another
successfully committing transaction, $B$, which arrived at the arbiter before $A$,
had $A$ in its list of predecessors ($A$ would then have been  added to the hit
list). That is, $B$ arrives in the system during the run time of $A$, tries to
update one of the edges that $A$ has updated, completes before $A$, goes to the
arbiter and is allowed to commit.

Suppose that $A$ contains $k$ updates, and let $t$ be the instant when the
$j$-th of those updates is attempted. The average interval from $t$ until the
completion of $A$, given that all updates succeed, is $(k+1-j)b$. If the $j$-th
update is of type $i$, let $h_i$ be the average interval from $t$ until the
completion of the next transaction, $B$, that updates the same edge and then
goes to the arbiter. That average can be estimated as
\begin{equation} \label{hi}
h_i = \frac{1}{\xi_i\alpha} + \frac{\kappa - r_1}{2(1-r_1)} b\;,
\end{equation}
where $\alpha$ is given by (\ref{al}). The multiplier of $b$ in the right-hand
side is half of the average number of updates in a transaction, given that
there are more than one.

Denote by $\beta_{ijk}$ the probability that $B$ arrives after the $j$-th
update out of the $k$ in $A$, and completes before $A$, and $A$ goes to the
arbiter but is aborted because $B$ commits, given that the $j$-th update is
of type $i$. We write
\begin{equation} \label{bijk}
\beta_{ijk} = \alpha (1-\beta )\frac{(k+1-j)b}{h_i + (k+1-j)b}\;.
\end{equation}

Removing the conditioning on the type of update, we get the probability,
$\beta_{jk}$, that $A$ is aborted by the arbiter due to the $j$-th of
its $k$ updates:
\begin{equation} \label{bjk}
\beta_{jk} = \sum_{i=1}^T \beta_{ijk}p_i\;.
\end{equation}

The probability, $\beta_{k}$, that at least one of the $k$ updates will
cause $A$ to be aborted, is
\begin{equation} \label{bk}
\beta_{k} = 1 - \prod_{j=1}^k (1-\beta_{jk})\;.
\end{equation}

Finally, the unconditional probability, $\beta$, that an arbitrary
transaction is aborted by the arbiter, can be expressed as
\begin{equation} \label{b}
\beta = \sum_{k=2}^\infty \beta_k r_k\;.
\end{equation}
The right-hand side of this equation depends on $\alpha$, which has already
been computed, and also on $\beta$. Thus, we have another fixed-point
equation which can be be solved by an iterative procedure of the type
(\ref{iter}).

One might wish to measure the performance of the system by a cost function
of the form
\begin{equation} \label{C}
C = c_1 W + c_2 R\;,
\end{equation}
where $c_1$ and $c_2$ are some coefficients reflecting the relative
importance given to the average response time and number of aborts. There
are trade-offs that may need to be controlled. If, for example, the arbiter
is overloaded, leading to large or infinite response times, a 'voluntary
abort' policy may be introduced. If a transaction cannot commit upon
completion (because its predecessor list is non-empty), it tosses a
biased coin and, with probability $\sigma$, aborts instead of going to
the arbiter. The offered load at the arbiter queue would then be reduced to
$\rho = \lambda\alpha (1-\sigma )s$. The optimal value of $\sigma$ would
be chosen so as to minimize the cost function $C$.

\section{Numerical and Simulation Results}

The purpose of this section is to assess the accuracy of the model estimates
by comparing them with simulations. In order to reduce the number of
parameters to be set, we focus on the smallest and most frequently accessed
class of edges, ignoring the larger classes where conflicts are very unlikely
to occur. The examples we have chosen contain a single class with $N$
distributed edges, each of which is equally likely to be the target of an
update. The size and traffic parameters are typical of a large scale-free graph
database (see also \cite{emw}).

In the first example, $N$ is varied between 5000 and 25000 edges. The arrival
rate is fixed at $\lambda =1000$ transactions per second. The average network
delay is assumed to be 5 milliseconds (i.e., $\delta = 200$). That is also the
value of $b$ (the average time per update). The distribution of the number of
updates in a transaction is geometric, with mean $\kappa =5$. The average
arbiter service time is $s=0.01$ and that value will be kept fixed in the
following examples.

In Figure \ref{fN}, the total average number, $R$, of transaction aborted per
unit time by the collision detection and by the order arbitration parts of the
protocol, is plotted against the number of edges. The estimated points are
computed by the algorithm described in section 3, while each simulated point
represents the result of a simulation run where one million transactions pass
through the system.

\begin{figure}[!ht]
\begin{center}
% GNUPLOT: LaTeX picture
\setlength{\unitlength}{0.22pt}
\ifx\plotpoint\undefined\newsavebox{\plotpoint}\fi
\sbox{\plotpoint}{\rule[-0.200pt]{0.400pt}{0.400pt}}%
\begin{picture}(1500,900)(0,0)
\sbox{\plotpoint}{\rule[-0.200pt]{0.400pt}{0.400pt}}%
\put(171.0,131.0){\rule[-0.200pt]{4.818pt}{0.400pt}}
\put(151,131){\makebox(0,0)[r]{ 0}}
\put(1429.0,131.0){\rule[-0.200pt]{4.818pt}{0.400pt}}
\put(171.0,222.0){\rule[-0.200pt]{4.818pt}{0.400pt}}
\put(151,222){\makebox(0,0)[r]{ 5}}
\put(1429.0,222.0){\rule[-0.200pt]{4.818pt}{0.400pt}}
\put(171.0,313.0){\rule[-0.200pt]{4.818pt}{0.400pt}}
\put(151,313){\makebox(0,0)[r]{ 10}}
\put(1429.0,313.0){\rule[-0.200pt]{4.818pt}{0.400pt}}
\put(171.0,404.0){\rule[-0.200pt]{4.818pt}{0.400pt}}
\put(151,404){\makebox(0,0)[r]{ 15}}
\put(1429.0,404.0){\rule[-0.200pt]{4.818pt}{0.400pt}}
\put(171.0,495.0){\rule[-0.200pt]{4.818pt}{0.400pt}}
\put(151,495){\makebox(0,0)[r]{ 20}}
\put(1429.0,495.0){\rule[-0.200pt]{4.818pt}{0.400pt}}
\put(171.0,586.0){\rule[-0.200pt]{4.818pt}{0.400pt}}
\put(151,586){\makebox(0,0)[r]{ 25}}
\put(1429.0,586.0){\rule[-0.200pt]{4.818pt}{0.400pt}}
\put(171.0,677.0){\rule[-0.200pt]{4.818pt}{0.400pt}}
\put(151,677){\makebox(0,0)[r]{ 30}}
\put(1429.0,677.0){\rule[-0.200pt]{4.818pt}{0.400pt}}
\put(171.0,768.0){\rule[-0.200pt]{4.818pt}{0.400pt}}
\put(151,768){\makebox(0,0)[r]{ 35}}
\put(1429.0,768.0){\rule[-0.200pt]{4.818pt}{0.400pt}}
\put(171.0,859.0){\rule[-0.200pt]{4.818pt}{0.400pt}}
\put(151,859){\makebox(0,0)[r]{ 40}}
\put(1429.0,859.0){\rule[-0.200pt]{4.818pt}{0.400pt}}
\put(171.0,131.0){\rule[-0.200pt]{0.400pt}{4.818pt}}
\put(171,90){\makebox(0,0){ 5000}}
\put(171.0,839.0){\rule[-0.200pt]{0.400pt}{4.818pt}}
\put(490.0,131.0){\rule[-0.200pt]{0.400pt}{4.818pt}}
\put(490,90){\makebox(0,0){ 10000}}
\put(490.0,839.0){\rule[-0.200pt]{0.400pt}{4.818pt}}
\put(810.0,131.0){\rule[-0.200pt]{0.400pt}{4.818pt}}
\put(810,90){\makebox(0,0){ 15000}}
\put(810.0,839.0){\rule[-0.200pt]{0.400pt}{4.818pt}}
\put(1129.0,131.0){\rule[-0.200pt]{0.400pt}{4.818pt}}
\put(1129,90){\makebox(0,0){ 20000}}
\put(1129.0,839.0){\rule[-0.200pt]{0.400pt}{4.818pt}}
\put(1449.0,131.0){\rule[-0.200pt]{0.400pt}{4.818pt}}
\put(1449,90){\makebox(0,0){ 25000}}
\put(1449.0,839.0){\rule[-0.200pt]{0.400pt}{4.818pt}}
\put(171.0,131.0){\rule[-0.200pt]{0.400pt}{160pt}}
\put(171.0,131.0){\rule[-0.200pt]{282pt}{0.400pt}}
\put(1449.0,131.0){\rule[-0.200pt]{0.400pt}{160pt}}
\put(171.0,859.0){\rule[-0.200pt]{282pt}{0.400pt}}
\put(50,495){\makebox(0,0){$R$}}
\put(810,29){\makebox(0,0){$N$}}
\put(1289,819){\makebox(0,0)[r]{Model estimates}}
\put(1309.0,819.0){\rule[-0.200pt]{24.090pt}{0.400pt}}
\put(171,696){\usebox{\plotpoint}}
\multiput(171.00,694.92)(0.564,-0.500){563}{\rule{0.551pt}{0.120pt}}
\multiput(171.00,695.17)(317.857,-283.000){2}{\rule{0.275pt}{0.400pt}}
\multiput(490.00,411.92)(1.706,-0.499){185}{\rule{1.462pt}{0.120pt}}
\multiput(490.00,412.17)(316.966,-94.000){2}{\rule{0.731pt}{0.400pt}}
\multiput(810.00,317.92)(3.415,-0.498){91}{\rule{2.815pt}{0.120pt}}
\multiput(810.00,318.17)(313.158,-47.000){2}{\rule{1.407pt}{0.400pt}}
\multiput(1129.00,270.92)(5.781,-0.497){53}{\rule{4.671pt}{0.120pt}}
\multiput(1129.00,271.17)(310.304,-28.000){2}{\rule{2.336pt}{0.400pt}}
\put(171,696){\makebox(0,0){$+$}}
\put(490,413){\makebox(0,0){$+$}}
\put(810,319){\makebox(0,0){$+$}}
\put(1129,272){\makebox(0,0){$+$}}
\put(1449,244){\makebox(0,0){$+$}}
\put(1359,819){\makebox(0,0){$+$}}
\put(1289,778){\makebox(0,0)[r]{Simulations}}
\multiput(1309,778)(20.756,0.000){5}{\usebox{\plotpoint}}
\put(1409,778){\usebox{\plotpoint}}
\put(171,752){\usebox{\plotpoint}}
\multiput(171,752)(14.608,-14.745){22}{\usebox{\plotpoint}}
\multiput(490,430)(19.846,-6.078){16}{\usebox{\plotpoint}}
\multiput(810,332)(20.475,-3.402){16}{\usebox{\plotpoint}}
\multiput(1129,279)(20.671,-1.873){16}{\usebox{\plotpoint}}
\put(1449,250){\usebox{\plotpoint}}
\put(171,752){\makebox(0,0){$\times$}}
\put(490,430){\makebox(0,0){$\times$}}
\put(810,332){\makebox(0,0){$\times$}}
\put(1129,279){\makebox(0,0){$\times$}}
\put(1449,250){\makebox(0,0){$\times$}}
\put(1359,778){\makebox(0,0){$\times$}}
\end{picture}
\caption{Abort rate as a function of $N$} \label{fN}
$\lambda =1000$, $\kappa =5$, $\delta = 200$, $b=0.005$, $s=0.01$
\end{center}
\end{figure}

Intuitively, we expect that when the number of edges increases, there will
be fewer collisions and instances of overtaking, and therefore fewer aborts.
Indeed, that is what is observed. The model consistently underestimates the
number of aborts, but the relative errors are not large. They vary from 9\%
at $N=5000$ to 5\% at $N=25000$. That underestimation is probably caused by
the simplifying assumptions used in deriving the approximate estimates. On
the other hand, the times taken to produce the two plots were vastly different:
the model plot took a small fraction of a second to compute, while the
simulation runs were several orders of magnitude slower.

From now on, the number of edges will be fixed at $N=10000$ and the effect of
different parameters will be explored. In the second example, the arrival rate
$\lambda$ is varied between 700 and 1200 transactions per second, while the other
parameters are kept as before.

\begin{figure}[!ht]
    \begin{center}
        % GNUPLOT: LaTeX picture
%		\setlength{\unitlength}{0.240900pt}
        \setlength{\unitlength}{0.22pt}
        \ifx\plotpoint\undefined\newsavebox{\plotpoint}\fi
        \sbox{\plotpoint}{\rule[-0.200pt]{0.400pt}{0.400pt}}%
        \begin{picture}(1500,900)(0,0)
        \sbox{\plotpoint}{\rule[-0.200pt]{0.400pt}{0.400pt}}%
        \put(171.0,131.0){\rule[-0.200pt]{4.818pt}{0.400pt}}
        \put(151,131){\makebox(0,0)[r]{ 0}}
        \put(1429.0,131.0){\rule[-0.200pt]{4.818pt}{0.400pt}}
        \put(171.0,277.0){\rule[-0.200pt]{4.818pt}{0.400pt}}
        \put(151,277){\makebox(0,0)[r]{ 5}}
        \put(1429.0,277.0){\rule[-0.200pt]{4.818pt}{0.400pt}}
        \put(171.0,422.0){\rule[-0.200pt]{4.818pt}{0.400pt}}
        \put(151,422){\makebox(0,0)[r]{ 10}}
        \put(1429.0,422.0){\rule[-0.200pt]{4.818pt}{0.400pt}}
        \put(171.0,568.0){\rule[-0.200pt]{4.818pt}{0.400pt}}
        \put(151,568){\makebox(0,0)[r]{ 15}}
        \put(1429.0,568.0){\rule[-0.200pt]{4.818pt}{0.400pt}}
        \put(171.0,713.0){\rule[-0.200pt]{4.818pt}{0.400pt}}
        \put(151,713){\makebox(0,0)[r]{ 20}}
        \put(1429.0,713.0){\rule[-0.200pt]{4.818pt}{0.400pt}}
        \put(171.0,859.0){\rule[-0.200pt]{4.818pt}{0.400pt}}
        \put(151,859){\makebox(0,0)[r]{ 25}}
        \put(1429.0,859.0){\rule[-0.200pt]{4.818pt}{0.400pt}}
        \put(171.0,131.0){\rule[-0.200pt]{0.400pt}{4.818pt}}
        \put(171,90){\makebox(0,0){ 700}}
        \put(171.0,839.0){\rule[-0.200pt]{0.400pt}{4.818pt}}
        \put(427.0,131.0){\rule[-0.200pt]{0.400pt}{4.818pt}}
        \put(427,90){\makebox(0,0){ 800}}
        \put(427.0,839.0){\rule[-0.200pt]{0.400pt}{4.818pt}}
        \put(682.0,131.0){\rule[-0.200pt]{0.400pt}{4.818pt}}
        \put(682,90){\makebox(0,0){ 900}}
        \put(682.0,839.0){\rule[-0.200pt]{0.400pt}{4.818pt}}
        \put(938.0,131.0){\rule[-0.200pt]{0.400pt}{4.818pt}}
        \put(938,90){\makebox(0,0){ 1000}}
        \put(938.0,839.0){\rule[-0.200pt]{0.400pt}{4.818pt}}
        \put(1193.0,131.0){\rule[-0.200pt]{0.400pt}{4.818pt}}
        \put(1193,90){\makebox(0,0){ 1100}}
        \put(1193.0,839.0){\rule[-0.200pt]{0.400pt}{4.818pt}}
        \put(1449.0,131.0){\rule[-0.200pt]{0.400pt}{4.818pt}}
        \put(1449,90){\makebox(0,0){ 1200}}
        \put(1449.0,839.0){\rule[-0.200pt]{0.400pt}{4.818pt}}
        \put(171.0,131.0){\rule[-0.200pt]{0.400pt}{160pt}}
        \put(171.0,131.0){\rule[-0.200pt]{282pt}{0.400pt}}
        \put(1449.0,131.0){\rule[-0.200pt]{0.400pt}{160pt}}
        \put(171.0,859.0){\rule[-0.200pt]{282pt}{0.400pt}}
        \put(50,495){\makebox(0,0){$R$}}
        \put(810,29){\makebox(0,0){$\lambda$}}
        \put(551,819){\makebox(0,0)[r]{Model estimates}}
        \put(571.0,819.0){\rule[-0.200pt]{24.090pt}{0.400pt}}
        \put(171,352){\usebox{\plotpoint}}
        \multiput(171.00,352.58)(1.889,0.499){133}{\rule{1.606pt}{0.120pt}}
        \multiput(171.00,351.17)(252.667,68.000){2}{\rule{0.803pt}{0.400pt}}
        \multiput(427.00,420.58)(1.683,0.499){149}{\rule{1.442pt}{0.120pt}}
        \multiput(427.00,419.17)(252.007,76.000){2}{\rule{0.721pt}{0.400pt}}
        \multiput(682.00,496.58)(1.492,0.499){169}{\rule{1.291pt}{0.120pt}}
        \multiput(682.00,495.17)(253.321,86.000){2}{\rule{0.645pt}{0.400pt}}
        \multiput(938.00,582.58)(1.359,0.499){185}{\rule{1.185pt}{0.120pt}}
        \multiput(938.00,581.17)(252.540,94.000){2}{\rule{0.593pt}{0.400pt}}
        \multiput(1193.00,676.58)(1.233,0.499){205}{\rule{1.085pt}{0.120pt}}
        \multiput(1193.00,675.17)(253.749,104.000){2}{\rule{0.542pt}{0.400pt}}
        \put(171,352){\makebox(0,0){$+$}}
        \put(427,420){\makebox(0,0){$+$}}
        \put(682,496){\makebox(0,0){$+$}}
        \put(938,582){\makebox(0,0){$+$}}
        \put(1193,676){\makebox(0,0){$+$}}
        \put(1449,780){\makebox(0,0){$+$}}
        \put(621,819){\makebox(0,0){$+$}}
        \put(551,778){\makebox(0,0)[r]{Simulations}}
        \multiput(571,778)(20.756,0.000){5}{\usebox{\plotpoint}}
        \put(671,778){\usebox{\plotpoint}}
        \put(171,361){\usebox{\plotpoint}}
        \multiput(171,361)(19.918,5.835){13}{\usebox{\plotpoint}}
        \multiput(427,436)(19.690,6.563){13}{\usebox{\plotpoint}}
        \multiput(682,521)(19.605,6.816){13}{\usebox{\plotpoint}}
        \multiput(938,610)(19.245,7.773){14}{\usebox{\plotpoint}}
        \multiput(1193,713)(19.043,8.257){13}{\usebox{\plotpoint}}
        \put(1449,824){\usebox{\plotpoint}}
        \put(171,361){\makebox(0,0){$\times$}}
        \put(427,436){\makebox(0,0){$\times$}}
        \put(682,521){\makebox(0,0){$\times$}}
        \put(938,610){\makebox(0,0){$\times$}}
        \put(1193,713){\makebox(0,0){$\times$}}
        \put(1449,824){\makebox(0,0){$\times$}}
        \put(621,778){\makebox(0,0){$\times$}}
%		\put(171.0,131.0){\rule[-0.200pt]{0.400pt}{175.375pt}}
%		\put(171.0,131.0){\rule[-0.200pt]{307.870pt}{0.400pt}}
%		\put(1449.0,131.0){\rule[-0.200pt]{0.400pt}{175.375pt}}
%		\put(171.0,859.0){\rule[-0.200pt]{307.870pt}{0.400pt}}
        \end{picture}
        \caption{Abort rate as a function of $\lambda$} \label{f2}
        $\kappa =5$, $\delta = 200$, $b=0.005$, $s=0.01$
    \end{center}
\end{figure}

In Figure \ref{f2}, the average number of aborted transactions per second, $R$,
is plotted against the arrival rate $\lambda$, using both the model approximation
and simulations. Each simulated point is again the result of a run where one
million transactions pass through the system. Once more, we observe that the model
slightly under-estimates the values of $R$, but the relative errors are quite small;
they are on the order of 6\% or less, over the entire range.

The average response time of a transaction, $W$, was about 25 milliseconds; its
value changed very little over this range of arrival rates.

For these parameter values, the model predicts that the arbiter queue becomes
unstable when the arrival rate is about $\lambda = 1500$. The simulation
agrees. The observed rate at which transactions join the arbiter queue exceeds
the service rate, $\mu = 100$, for that value of $\lambda$.

For the next experiment, the average network delay is doubled to 10 milliseconds,
$\delta = 100$. Intuitively, this should have the effect of increasing the rate
at which transactions are aborted, and also should increase the offered load at
the arbiter queue.

\begin{figure}[!ht]
    \begin{center}
% GNUPLOT: LaTeX picture
\setlength{\unitlength}{0.22pt}
\ifx\plotpoint\undefined\newsavebox{\plotpoint}\fi
\sbox{\plotpoint}{\rule[-0.200pt]{0.400pt}{0.400pt}}%
\begin{picture}(1500,900)(0,0)
\sbox{\plotpoint}{\rule[-0.200pt]{0.400pt}{0.400pt}}%
\put(171.0,131.0){\rule[-0.200pt]{4.818pt}{0.400pt}}
\put(151,131){\makebox(0,0)[r]{ 0}}
\put(1429.0,131.0){\rule[-0.200pt]{4.818pt}{0.400pt}}
\put(171.0,222.0){\rule[-0.200pt]{4.818pt}{0.400pt}}
\put(151,222){\makebox(0,0)[r]{ 5}}
\put(1429.0,222.0){\rule[-0.200pt]{4.818pt}{0.400pt}}
\put(171.0,313.0){\rule[-0.200pt]{4.818pt}{0.400pt}}
\put(151,313){\makebox(0,0)[r]{ 10}}
\put(1429.0,313.0){\rule[-0.200pt]{4.818pt}{0.400pt}}
\put(171.0,404.0){\rule[-0.200pt]{4.818pt}{0.400pt}}
\put(151,404){\makebox(0,0)[r]{ 15}}
\put(1429.0,404.0){\rule[-0.200pt]{4.818pt}{0.400pt}}
\put(171.0,495.0){\rule[-0.200pt]{4.818pt}{0.400pt}}
\put(151,495){\makebox(0,0)[r]{ 20}}
\put(1429.0,495.0){\rule[-0.200pt]{4.818pt}{0.400pt}}
\put(171.0,586.0){\rule[-0.200pt]{4.818pt}{0.400pt}}
\put(151,586){\makebox(0,0)[r]{ 25}}
\put(1429.0,586.0){\rule[-0.200pt]{4.818pt}{0.400pt}}
\put(171.0,677.0){\rule[-0.200pt]{4.818pt}{0.400pt}}
\put(151,677){\makebox(0,0)[r]{ 30}}
\put(1429.0,677.0){\rule[-0.200pt]{4.818pt}{0.400pt}}
\put(171.0,768.0){\rule[-0.200pt]{4.818pt}{0.400pt}}
\put(151,768){\makebox(0,0)[r]{ 35}}
\put(1429.0,768.0){\rule[-0.200pt]{4.818pt}{0.400pt}}
\put(171.0,859.0){\rule[-0.200pt]{4.818pt}{0.400pt}}
\put(151,859){\makebox(0,0)[r]{ 40}}
\put(1429.0,859.0){\rule[-0.200pt]{4.818pt}{0.400pt}}
\put(171.0,131.0){\rule[-0.200pt]{0.400pt}{4.818pt}}
\put(171,90){\makebox(0,0){ 600}}
\put(171.0,839.0){\rule[-0.200pt]{0.400pt}{4.818pt}}
\put(331.0,131.0){\rule[-0.200pt]{0.400pt}{4.818pt}}
\put(331,90){\makebox(0,0){ 650}}
\put(331.0,839.0){\rule[-0.200pt]{0.400pt}{4.818pt}}
\put(490.0,131.0){\rule[-0.200pt]{0.400pt}{4.818pt}}
\put(490,90){\makebox(0,0){ 700}}
\put(490.0,839.0){\rule[-0.200pt]{0.400pt}{4.818pt}}
\put(650.0,131.0){\rule[-0.200pt]{0.400pt}{4.818pt}}
\put(650,90){\makebox(0,0){ 750}}
\put(650.0,839.0){\rule[-0.200pt]{0.400pt}{4.818pt}}
\put(810.0,131.0){\rule[-0.200pt]{0.400pt}{4.818pt}}
\put(810,90){\makebox(0,0){ 800}}
\put(810.0,839.0){\rule[-0.200pt]{0.400pt}{4.818pt}}
\put(970.0,131.0){\rule[-0.200pt]{0.400pt}{4.818pt}}
\put(970,90){\makebox(0,0){ 850}}
\put(970.0,839.0){\rule[-0.200pt]{0.400pt}{4.818pt}}
\put(1129.0,131.0){\rule[-0.200pt]{0.400pt}{4.818pt}}
\put(1129,90){\makebox(0,0){ 900}}
\put(1129.0,839.0){\rule[-0.200pt]{0.400pt}{4.818pt}}
\put(1289.0,131.0){\rule[-0.200pt]{0.400pt}{4.818pt}}
\put(1289,90){\makebox(0,0){ 950}}
\put(1289.0,839.0){\rule[-0.200pt]{0.400pt}{4.818pt}}
\put(1449.0,131.0){\rule[-0.200pt]{0.400pt}{4.818pt}}
\put(1449,90){\makebox(0,0){ 1000}}
\put(1449.0,839.0){\rule[-0.200pt]{0.400pt}{4.818pt}}
\put(171.0,131.0){\rule[-0.200pt]{0.400pt}{160pt}}
\put(171.0,131.0){\rule[-0.200pt]{282pt}{0.400pt}}
\put(1449.0,131.0){\rule[-0.200pt]{0.400pt}{160pt}}
\put(171.0,859.0){\rule[-0.200pt]{282pt}{0.400pt}}
\put(50,495){\makebox(0,0){$R$}}
\put(810,29){\makebox(0,0){$\lambda$}}
\put(551,819){\makebox(0,0)[r]{Model estimates}}
\put(571.0,819.0){\rule[-0.200pt]{24.090pt}{0.400pt}}
\put(171,334){\usebox{\plotpoint}}
\multiput(171.00,334.58)(2.192,0.499){143}{\rule{1.848pt}{0.120pt}}
\multiput(171.00,333.17)(315.164,73.000){2}{\rule{0.924pt}{0.400pt}}
\multiput(490.00,407.58)(1.910,0.499){165}{\rule{1.624pt}{0.120pt}}
\multiput(490.00,406.17)(316.630,84.000){2}{\rule{0.812pt}{0.400pt}}
\multiput(810.00,491.58)(1.665,0.499){189}{\rule{1.429pt}{0.120pt}}
\multiput(810.00,490.17)(316.034,96.000){2}{\rule{0.715pt}{0.400pt}}
\multiput(1129.00,587.58)(1.498,0.499){211}{\rule{1.296pt}{0.120pt}}
\multiput(1129.00,586.17)(317.310,107.000){2}{\rule{0.648pt}{0.400pt}}
\put(171,334){\makebox(0,0){$+$}}
\put(490,407){\makebox(0,0){$+$}}
\put(810,491){\makebox(0,0){$+$}}
\put(1129,587){\makebox(0,0){$+$}}
\put(1449,694){\makebox(0,0){$+$}}
\put(621,819){\makebox(0,0){$+$}}
\put(551,778){\makebox(0,0)[r]{Simulations}}
\multiput(571,778)(20.756,0.000){5}{\usebox{\plotpoint}}
\put(671,778){\usebox{\plotpoint}}
\put(171,350){\usebox{\plotpoint}}
\multiput(171,350)(20.087,5.226){16}{\usebox{\plotpoint}}
\multiput(490,433)(19.931,5.792){16}{\usebox{\plotpoint}}
\multiput(810,526)(19.715,6.489){17}{\usebox{\plotpoint}}
\multiput(1129,631)(19.513,7.073){16}{\usebox{\plotpoint}}
\put(1449,747){\usebox{\plotpoint}}
\put(171,350){\makebox(0,0){$\times$}}
\put(490,433){\makebox(0,0){$\times$}}
\put(810,526){\makebox(0,0){$\times$}}
\put(1129,631){\makebox(0,0){$\times$}}
\put(1449,747){\makebox(0,0){$\times$}}
\put(621,778){\makebox(0,0){$\times$}}
\end{picture}
        \caption{Larger network delays} \label{f3}
        $\kappa =5$, $\delta = 100$, $b=0.01$, $s=0.01$
    \end{center}
\end{figure}

Figure \ref{f3} confirms our intuition. The relative errors of the model
estimates are still quite low, on the order of 9\% or less. The arrival
rate is now varied between $\lambda =600$ and $\lambda =1000$. Both the
model and the simulation agree that the arbiter queue becomes unstable
when $\lambda =1100$.

In the fourth experiment, the network delay is back to 5 milliseconds, but
the number of updates in a transaction, $K$, has a different distribution
and mean. The assumption now is that $K$ is uniformly distributed on the
range [1,19], with a mean of 10. The results are illustrated in Figure
\ref{f4}

\begin{figure}[!ht]
    \begin{center}
% GNUPLOT: LaTeX picture
\setlength{\unitlength}{0.22pt}
\ifx\plotpoint\undefined\newsavebox{\plotpoint}\fi
\sbox{\plotpoint}{\rule[-0.200pt]{0.400pt}{0.400pt}}%
\begin{picture}(1500,900)(0,0)
\sbox{\plotpoint}{\rule[-0.200pt]{0.400pt}{0.400pt}}%
\put(171.0,131.0){\rule[-0.200pt]{4.818pt}{0.400pt}}
\put(151,131){\makebox(0,0)[r]{ 0}}
\put(1429.0,131.0){\rule[-0.200pt]{4.818pt}{0.400pt}}
\put(171.0,277.0){\rule[-0.200pt]{4.818pt}{0.400pt}}
\put(151,277){\makebox(0,0)[r]{ 5}}
\put(1429.0,277.0){\rule[-0.200pt]{4.818pt}{0.400pt}}
\put(171.0,422.0){\rule[-0.200pt]{4.818pt}{0.400pt}}
\put(151,422){\makebox(0,0)[r]{ 10}}
\put(1429.0,422.0){\rule[-0.200pt]{4.818pt}{0.400pt}}
\put(171.0,568.0){\rule[-0.200pt]{4.818pt}{0.400pt}}
\put(151,568){\makebox(0,0)[r]{ 15}}
\put(1429.0,568.0){\rule[-0.200pt]{4.818pt}{0.400pt}}
\put(171.0,713.0){\rule[-0.200pt]{4.818pt}{0.400pt}}
\put(151,713){\makebox(0,0)[r]{ 20}}
\put(1429.0,713.0){\rule[-0.200pt]{4.818pt}{0.400pt}}
\put(171.0,859.0){\rule[-0.200pt]{4.818pt}{0.400pt}}
\put(151,859){\makebox(0,0)[r]{ 25}}
\put(1429.0,859.0){\rule[-0.200pt]{4.818pt}{0.400pt}}
\put(171.0,131.0){\rule[-0.200pt]{0.400pt}{4.818pt}}
\put(171,90){\makebox(0,0){ 400}}
\put(171.0,839.0){\rule[-0.200pt]{0.400pt}{4.818pt}}
\put(384.0,131.0){\rule[-0.200pt]{0.400pt}{4.818pt}}
\put(384,90){\makebox(0,0){ 420}}
\put(384.0,839.0){\rule[-0.200pt]{0.400pt}{4.818pt}}
\put(597.0,131.0){\rule[-0.200pt]{0.400pt}{4.818pt}}
\put(597,90){\makebox(0,0){ 440}}
\put(597.0,839.0){\rule[-0.200pt]{0.400pt}{4.818pt}}
\put(810.0,131.0){\rule[-0.200pt]{0.400pt}{4.818pt}}
\put(810,90){\makebox(0,0){ 460}}
\put(810.0,839.0){\rule[-0.200pt]{0.400pt}{4.818pt}}
\put(1023.0,131.0){\rule[-0.200pt]{0.400pt}{4.818pt}}
\put(1023,90){\makebox(0,0){ 480}}
\put(1023.0,839.0){\rule[-0.200pt]{0.400pt}{4.818pt}}
\put(1236.0,131.0){\rule[-0.200pt]{0.400pt}{4.818pt}}
\put(1236,90){\makebox(0,0){ 500}}
\put(1236.0,839.0){\rule[-0.200pt]{0.400pt}{4.818pt}}
\put(1449.0,131.0){\rule[-0.200pt]{0.400pt}{4.818pt}}
\put(1449,90){\makebox(0,0){ 520}}
\put(1449.0,839.0){\rule[-0.200pt]{0.400pt}{4.818pt}}
\put(171.0,131.0){\rule[-0.200pt]{0.400pt}{160pt}}
\put(171.0,131.0){\rule[-0.200pt]{282pt}{0.400pt}}
\put(1449.0,131.0){\rule[-0.200pt]{0.400pt}{160pt}}
\put(171.0,859.0){\rule[-0.200pt]{282pt}{0.400pt}}
\put(50,495){\makebox(0,0){$R$}}
\put(810,29){\makebox(0,0){$\lambda$}}
\put(551,819){\makebox(0,0)[r]{Model estimates}}
\put(571.0,819.0){\rule[-0.200pt]{24.090pt}{0.400pt}}
\put(171,434){\usebox{\plotpoint}}
\multiput(171.00,434.58)(3.218,0.498){97}{\rule{2.660pt}{0.120pt}}
\multiput(171.00,433.17)(314.479,50.000){2}{\rule{1.330pt}{0.400pt}}
\multiput(491.00,484.58)(2.915,0.499){107}{\rule{2.420pt}{0.120pt}}
\multiput(491.00,483.17)(313.977,55.000){2}{\rule{1.210pt}{0.400pt}}
\multiput(810.00,539.58)(2.679,0.499){117}{\rule{2.233pt}{0.120pt}}
\multiput(810.00,538.17)(315.365,60.000){2}{\rule{1.117pt}{0.400pt}}
\multiput(1130.00,599.58)(2.287,0.499){137}{\rule{1.923pt}{0.120pt}}
\multiput(1130.00,598.17)(315.009,70.000){2}{\rule{0.961pt}{0.400pt}}
\put(171,434){\makebox(0,0){$+$}}
\put(491,484){\makebox(0,0){$+$}}
\put(810,539){\makebox(0,0){$+$}}
\put(1130,599){\makebox(0,0){$+$}}
\put(1449,669){\makebox(0,0){$+$}}
\put(621,819){\makebox(0,0){$+$}}
\put(551,778){\makebox(0,0)[r]{Simulations}}
\multiput(571,778)(20.756,0.000){5}{\usebox{\plotpoint}}
\put(671,778){\usebox{\plotpoint}}
\put(171,459){\usebox{\plotpoint}}
\multiput(171,459)(20.507,3.204){16}{\usebox{\plotpoint}}
\multiput(491,509)(20.386,3.898){16}{\usebox{\plotpoint}}
\multiput(810,570)(20.445,3.578){15}{\usebox{\plotpoint}}
\multiput(1130,626)(20.299,4.327){16}{\usebox{\plotpoint}}
\put(1449,694){\usebox{\plotpoint}}
\put(171,459){\makebox(0,0){$\times$}}
\put(491,509){\makebox(0,0){$\times$}}
\put(810,570){\makebox(0,0){$\times$}}
\put(1130,626){\makebox(0,0){$\times$}}
\put(1449,694){\makebox(0,0){$\times$}}
\put(621,778){\makebox(0,0){$\times$}}
\end{picture}
        \caption{Different distribution of updates} \label{f4}
        $\kappa =10$, $\delta = 200$, $b=0.005$, $s=0.01$
    \end{center}
\end{figure}

The larger number of updates per transaction leads to both higher
likelihood of collisions and more visits to the arbiter. The
saturation point for the arbiter queue is now a little below
$\lambda =550$. As Figure \ref{f4} illustrates, the model
approximation is still accurate, with relative errors on the
order of 8\% or less.

It is perhaps worth noting that in the last three examples, the rate
of aborts increases roughly linearly with $\lambda$. For all arrival
rates in example 2, between 1\% and 2\% of the incoming transactions
are aborted. In example 3 that fraction is between 2\% and 3\%, while
in example 4 it is between 3\% and 4\%.

\section{Conclusion}

We have addressed the information corruption problem caused by interferences
among transactions which update distributed edges. The proposed concurrency control
protocol has two distinct mechanisms: collision detection and
arbitration between transactions. That protocol has an impact on system performance,
in terms of aborted transactions and load on the arbiter. To evaluate this impact, an
approximate model was developed and solved. It provides estimates for the average
number of transactions that are aborted per unit time, the probability that a
transaction will need to go to arbitration, and the average response time of a
transaction. The accuracy of the solution was examined by comparisons with simulations
and was found to be very high under a variety of parameter settings.

Similar to the edge-order inconsistency examined here, there may also
be {\em node-order} inconsistency, occuring when transactions interfere
while updating the same set of nodes. Eliminating node-order inconsistencies
will be addressed in future work. It is well known in the database literature
that there is a hierarchy of approaches which achieve various degrees of
concurrency control (see \cite{ady}). Selecting an approach for a given
application typically involves a trade-off between consistency requirements
and performance. The most stringent common form of concurrency control is
{\em serializability}, which maintains an abstraction of transactions being
executed in some serial order. That requirement incurs the highest performance
overhead.

\begin{thebibliography}{99}
\bibitem{ady} Adya, A.: Weak consistency: a generalized theory and optimistic
implementations for distributed transactions. PhD thesis, Massachusetts
Institute of Technology (1999)
\bibitem{bai} Bailis, P. and Ghodsi, A.: Eventual Consistency Today:
  Limitations, Extensions, and Beyond. ACMQueue \textbf{11}(3), 20--32 (2013)
\bibitem{cas} Apache Cassandra, \url{http://cassandra.apache.org/}. Last accessed 11 Dec 2019
\bibitem{dec} DeCandia, D. Hastorun, D., Jampani, M., Kakulapati, G.,
Lakshman, A., Pilchin, A., Sivasubramanian, S., Vosshall, P. and Vogels, W.:
Dynamo: Amazon's Highly Available Key-value Store. SIGOPS Oper. Syst. Rev.
\textbf{41}(6), 205--220 (2007)
\bibitem{emw} Ezhilchelvan P., Mitrani I., and Webber J. (2018). On the degradation of distributed graph databases with eventual consistency. In: Bakhshi R., Ballarini P., Barbot B., Castel-Taleb H., Remke A. (eds.) EUROPEAN PERFORMANCE ENGINEERING WORKSHOP 2018, LNCS, vol. 11178, pp. 1--13. Springer, Cham (2018)
\bibitem{occ} Kung, H.T. and Robinson, J.T.: On optimistic methods for
  concurrency control. ACM Trans. on Database Systems (TODS) \textbf{6}(2), 213--226 (1981)
\bibitem{paxos} Lamport, L.: The part-time parliament. ACM Trans. on Computer
  Systems (TOCS) \textbf{16}(2) 133--169 (1998)
\bibitem{raft} Ongaro, D., and Ousterhout, J.: In search of an understandable
  consensus algorithm. In: USENIX Annual Technical Conference, pp. 305--319. USENIX Association, Philadelphia, PA (2014)
\bibitem{warp} Escriva, R., Wong, B. and Sirer, E.G.: Warp: Lightweight Multi-Key
Transactions for Key-Value Stores. CoRR:abs/1509.07815, (2015)
\bibitem{hua} Huang, J. and Abadi, D.J.: Leopard: lightweight edge-oriented
partitioning and replication for dynamic graphs. VLDB Endowment \textbf{9}(7), 40--551 (2016)
\bibitem{fir} Firth, H. and Missier, P.: TAPER: query-aware, partition-enhancement
for large, heterogeneous graphs. Distributed and Parallel Databases \textbf{35}(2), 85--115 (2017)
\bibitem{rob} Robinson, I., Webber, J. and Eifrem, E.: Graph Databases, New
Opportunities for Connected Data. O'Reilly Media, Inc (2015)
\bibitem{sta} Stanton, I. and Kliot, G.: Streaming Graph Partitioning for Large
Distributed Graphs. In: 18th ACM SIGKDD Int. Conf. on Knowledge Discovery
and Data Mining, pp. 1222--1230, ACM, Beijing, China (2012)
\bibitem{vog} Vogels, W.: Eventually Consistent. Comm. ACM, \textbf{52}(1), 40--44 (2009)
\end{thebibliography}

\end{document}
